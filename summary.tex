\section{Summary}
\label{sec:summary}

We have discussed two of the three aspects one needs to take into account
to simulate the ICD spectrum of rare gas clusters in detail. Due to the nature
of clusters, a neon atom ionized in the inner-valence will have several decay
partners at different distances. The larger the interatomic distance the higher
is the kinetic energy of the ICD electrons which will yield a multitude of
peaks in the spectrum. These are then weighted by the decay width, which
depends on the interatomic distance of the decay partners but also needs
to be scaled by the number of pairs of the same distance.
The manifold of all different decay events will then yield the spectrum.

When applying these apsects to cluster structures, one finds that not only
the nearest neighbours contribute to the spectrum, but several other decay
partners do as well. Decay partners until a distance of at least twice the
distance of the clostest decay partners should be taken into account.
In clusters with an icosahedral cluster structure this is especially important
because the smallest interatomic distance between atoms of the same and atoms
of different layers is different, but both distances are comparable.
This leads to a different number of peaks in the spectra which might help to
differentiate between clusters of icosahedral and cuboctahedral cluster structure.
