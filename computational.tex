\section{Computational Details}
The cluster structures used were constructed to have an ideal icosahedral or
face-centered-cubic geometry with optional additional incomplete outermost
shells. These are based on the van der Waals radii for neon
$r_{Ne}=\unit[1.54]{\AA}$ \cite{}
and argon $r_{Ar}=\unit[1.88]{\AA}$ \cite{}. In case of the NeAr clusters those
two cluster structures from Ref. \cite{Fasshauer14_1} were used that matched best
with the combination of the argon content in the cluster, the NeAr-ICD to total
ICD ratio and the peak position of the NeAr-ICD peak. These are xyz
for set 
and xyz for set.


\begin{table}[h]
 \caption{Experimental values for the single ionization potentials
          \cite{Fasshauer14_1}
          used for the estimation of the decay widths.}
 \label{table:exp_input}
 \centering
 \begin{tabular}{lc}
  \toprule
  indicator            &  value \\
  \midrule
  SIP(Ne2s)            &  \unit[47.75]{eV} \\
  SIP(Ne2p)            &  \unit[21.10]{eV} \\
  SIP(Ar3p)$_{c<3}$    &  \unit[15.40]{eV} \\
  SIP(Ar3p)$_{c\ge 3}$ &  \unit[15.20]{eV} \\
  \bottomrule
 \end{tabular}
\end{table}


The calculations to obtain the ICD electron spectra of those cluster
structures were performed with
the program HARDRoC \cite{HARDRoC} using experimental ionization energies shown
in Table \ref{table:exp_input} and curves fitted to the decay width of the
NeNe ICD of Ref. \cite{}.
