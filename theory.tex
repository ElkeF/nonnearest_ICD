\section{Theoretical Background}
\label{sec:theory}

In roder to simulate the ICD electron spectra for a given cluster structure,
the kinetic energies of the ICD electrons $E_{ICD}$ and the corresponding decay
decay widths of all pairs have to be determined. The ICD electron energies
are given by the differences between the initial state and the final state
energies $E_{in}$ and $E_{fin}$, respectively. The initial state energy
is given by the single ionization potential (SIP) of the sub-valence electron
of the entire system
and the final state energy is given by the double ionization potential (DIP).
In the asymptotic limit, which is a reasonably good approximation for
weakly bound systems, he initial state energy is approximated by the SIP
of the initially ionized unit $X_{in}$ and the final state energy can be
approximated
by the sum over the SIPs of the electron donating unit $X_D$ and the electron
emitting unit $X_E$ ionized in the final state as well as the
Coulomb repulsion between two point charges at the interatomic distance $R$

\begin{align}
 E_{ICD}^\beta &= E_{in}^\beta - E_{fin}^\beta \label{equation:E_sec}\\
 E_{in}        &= SIP(X_{in}) \label{equation:E_in}\\          
 E_{fin}^\beta &= SIP(X_{D}^\beta) + SIP(X_{E}^\beta) + \frac 1R
           \label{equation:E_fin}                . 
\end{align}
Here, $\beta$ denotes the selected decay channel.

Following Wentzel \cite{Wentzel27}, Feshbach\cite{Feshbach58,Feshbach62}
and Fano \cite{Fano61} the decay width is given by

\begin{equation}
 \Gamma_{\beta}(E_{res}) = 2\pi \left|
                           \braket{\Phi_{in}| H_f |\chi_{\beta}}
                           \right|^2   .
\end{equation}
Here, the bound and ionized initial state is described by $\ket{\Phi_{in}}$ and
the final continuum state of a particular decay channel $\beta$ is
given by $\ket{\chi_{\beta}}$.
Its challenging description involving both bound and continuum states can
amongst others be achieved by the FanoADC-Stieltjes approach, where a
subset of the $2h1p$ functions within the Algebraic Diagrammatic Construction
(ADC) are used to mimic the final state function $\ket{\chi_{\beta}}$.
By these means calculated discrete energies and corresponding transitions
moments are then used to construct a continuous function, which is evaluated
at the resonance energy approximated by the single ionization potential
corresponding to the initial state $E_{in}$. For a more detailed
description of the method and comparison to other approaches see References
\cite{Averbukh05,Fasshauer15_1} and references therein.
