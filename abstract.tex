\begin{abstract}
Interatomic Coulombic Decay (ICD) is an electronic decay process after
non-outer-valence ionization. It has been shown to occur in a multitude of small
and large systems. We discuss the effects of more than one possible decay
partner in detail using NeAr clusters and pure Ne clusters as examples.
The mostly underestimated contribution of decay with non-nearest neighbours
is highlighted.

We study the possibility to investigate the transition from
small clusters to the solid state by using the ICD electron spectra to
distinguish between icosahedral and cuboctahedral cluster structures.
\end{abstract}
