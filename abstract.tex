\begin{abstract}
Interatomic Coulombic Decay (ICD) is an electronic decay process after
non-outer-valence ionization. It has been shown to occur in a multitude of small
and large systems.
The effects of more than one possible decay
partner are discussed in detail using NeAr clusters
and pure Ne clusters as examples.
Hereby, the mostly underestimated contribution of decay with
non-nearest neighbours is highlighted. In the neon atoms, the lifetime of the
bulk atoms is found to be in excellent agreement with experiment, while the
lifetimes of the surface atoms differ significantly. Hence, the experimental
lifetime can not purely be explained by the effect of the number of
neighbours.

We propose the possibility to investigate the transition from
small clusters to the solid state by using the ICD electron spectra to
distinguish between icosahedral and cuboctahedral cluster structures.
\end{abstract}
