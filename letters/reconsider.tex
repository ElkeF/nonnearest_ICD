\documentclass[DIN,10pt,pagenumber=false,parskip=half,fromalign=left,fromphone=false,fromemail=true,fromurl=false,fromlogo=true,fromrule=false]{scrlttr2}
\usepackage[utf8]{inputenc}
\usepackage{ngerman}
\usepackage{units}
\usepackage{tabularx}
\usepackage[right]{eurosym}
\usepackage{graphicx}
\usepackage[top=2cm,bottom=0cm,left=2cm,right=2cm]{geometry}
%\RequirePackage{graphicx}

\setkomavar{fromname}{Dr. Elke Faßhauer}
\setkomavar{fromaddress}{CTCC\\ University of Tromsø{}
                         --- The Arctic University of Norway
                         \\ 9037 Tromsø \\ Norway}
%\setkomavar{fromphone}{06221/545220}
\setkomavar{fromemail}{elke.fasshauer@uit.no}
\setkomavar{subject}{Manuscript NJP-104327}
\setkomavar{signature}{Elke Faßhauer -- author of NJP-104327}
\setkomavar{fromlogo}{\includegraphics[width=3cm]{uit.pdf}}


\begin{document}

\begin{letter}{New Journal of Physics\\Editorial Board}
\textheight200mm
\opening{Dear Editor,}

My manuscript NJP-104327 has been refereed by a single Referee, whose very
short and negative comment lead to its rejection. As I will discuss in the
following, all criticism of the Referee is unsubstantiated.

The Referee critisized two aspects of my manuscript: 1. it being mostly directed
towards the ICD community and 2. its topic being too subtle and not conceptually
new.
The manuscript is directed towards the ICD community, but it is also
of interest
to the cluster and condensed matter communities as well as indirectly to cancer
research. The Referee's criticism 1. is therefore factually wrong.
The generalization of the descriptions of non-nearest neighbour
ICD is therefore not a subtle topic in itself. 
Peaks in the ICD spectra of clusters intrinsically carry
information about the cluster structure. This fact has not been investigated
previously and represents an important progress in the field of ICD.
Hence, the Referee's criticism 2. is also unsubstantiated.

One of the topics which presently attract vital attention is
to describe the DNA damage following an ICD in
water molecules around it. Obviously, in this setting several water molecules
at different distances are involved. Understanding the effect of non-nearest
neighbour ICD is crucial for the theoretical description of the process and the
interpretation of experimental spectra. My manuscript provides a basis for
this research. In order to make this more clear I added two sentences
stating this explicitly in the introduction:\\
\emph{Especially in the biological systems, in which ICD is discussed to play an
important role [20,21,22]
the DNA is surrounded by multiple moving
water molecules which can act as decay partners. Therefore, understanding the
effect of non-nearest neighbour decay will be crucial to study these systems.}
However, this feature of non-nearest neighbour ICD
has so far not been addressed by itself and we will fill the gap in this
paper. \dots

I am convinced that my article is a valuable contribution for the
readership of New Journal of Physics, highly appreciated by scientists
considering Interatomic Coulombic Decay (ICD),
as well as the transition from atomic and molecular clusters to the solid state
and even the causes of cancer.

I kindly ask you to further consider my work for publication in the New
Journal of Physics and to request another Referee's opinion.


\closing{Yours sincerely,}
\end{letter}

\end{document}
